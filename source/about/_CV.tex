\documentclass[utf8x,helvetica,narrow,english,nologo,notitle,totpages, booktabs]{europecv}
\usepackage[T1]{fontenc}
\usepackage{textcomp}
\usepackage{graphicx}
\usepackage[english]{babel}
\usepackage{geometry}
\geometry{verbose,a4paper,tmargin=1.27cm,bmargin=2cm,lmargin=1cm,rmargin=1cm}
\usepackage{microtype}
\usepackage{hyperref}
\usepackage{mdwlist}

\hypersetup{colorlinks=true, urlcolor=blue}
\renewcommand{\ttdefault}{phv}

\ecvname{Dario Bertini}
\ecvtelephone{+44 7960454417}
\ecvemail{berdario@gmail.com}

\begin{document}
\begin{europecv}
  \ecvpersonalinfo
  \ecvsection{Work experience}
  \ecvitem{}{References available upon request.\\}

  \ecvitem{November 2014 - ...}{NCCGroup - Security Consultant}
  \ecvitem{}{Working at NCCGroup bolstered my knowledge of network protocols, web apps security, linux systems and windows domains security thanks to the extensive internal training. After the training I developed some tools for internal use and I was tasked with performing security assessments for multinational companies. Such assessments covered a range of technologies (.NET, Python, PHP, etc.) and had elements of both white box code reviews and black box tests. XSS, CSRF, cleartext storage of passwords, path traversal and remote code execution are among the kind of vulnerability that I found and documented.\\}

  \ecvitem{September 2014 - October 2014}{CommentBubble - Software developer}
  \ecvitem{}{Setup the project into a proper Python module, as a building block for testing and continuous integration. Discovered an Insecure Direct Object Reference vulnerability and wrote tests to prevent its regression.\\}

  \ecvitem{February 2014 - October 2014}{Audiencerate - Software developer/devops}
  \ecvitem{}{Managed the provisioning of the system, which has a load of about 100 req/s. Wrote some simple integration/acceptance tests, Added monitoring into the system, rewrote the code dealing with the database and tested all the software before each production deployment. (Technologies used: Ansible, Vagrant, Selenium, Clojure, Python)\\}

  \ecvsection{Personal skills and competences}
  \ecvitem{Programming Languages}{
\begin{itemize*}
\item Python: Advanced knowledge. Used for web development, dealing with binary formats, scripting as well as cpu-intensive loads, wrapping native libraries with ctypes, sketching out some proof of concept gui. I also read the internal details of some Python implementations and wrote my own simple tool to manage virtual environments
\item Clojure: Used on and off since 2013, I'm comfortable with the language and the surrounding tooling.
\item Haskell: Good knowledge, but never used for huge projects
\item Java: Good knowledge and experience with legacy Java codebases (no tests, huge amounts of code duplication and accidental complexity).
\item Ruby: Contributed some simple patches upstream (e.g.\url{http://git.io/W7e-9w} \url{http://git.io/7kM4Gw})
\item Working knowledge of F\#, Ocaml, Javascript, Scala, C, Vala, C#, SQL.
\end{itemize*}
}
  \ecvitem[15pt]{Software Tools and Libraries}{DVCS (Bazaar, Mercurial, Git, Darcs), Continuous Integration, Package/Dependencies Manager/Build Systems, Bug
Trackers, I18n and L10n toolchain, IDEs/Editors, DB/Data Stores
(Postgresql, Mongodb, Redis), OpenGL, MapReduce, Documentation (Sphinx, Markdown),
Configuration Management and Deployment (Ansible), Emscripten, Powershell, Lucene, JNI, Jython}
  \ecvitem[15pt]{Other Computer competences}{Experienced with Linux (and with Windows and MacOSX to a lesser extent), contributed/maintained some packages for the Nixos Linux distribution\\
    & For further details I have some repositories online at: \url{https://github.com/berdario}}

  \ecvsection{Education and training}
  \ecvitem{}{Bachelor Degree in Computer Engineering}
  \ecvitem{Subjects covered}{Network security, Languages and compilers, Information theory, C++ and Java programming courses, Databases, Operating systems, Theoretical computer science,  Project Management, Artificial Intelligence, Information Retrieval, Stochastic Models.\\
& Thesis Work: Adaptation of a georeferentiation search application for a client-server architecture}
  \ecvitem[15pt]{Name of organisation providing education and training}{Università degli Studi di Bergamo}

  \ecvitem{}{Erasmus study period in Computer Science at University of Southern Denmark\\
& Subjects: String Algorithms, Programming Languages, Cloud Computing, Advanced Data Structures, Online Algorithms, Combinatorial Optimization}
\\
\\
  \ecvmothertongue[10pt]{Italian}
  \ecvlanguageheader{(*)}
  \ecvlanguage{English}{\ecvCOne}{\ecvCTwo}{\ecvCOne}{\ecvCOne}{\ecvCOne}
  \ecvlastlanguage{Danish}{\ecvAOne}{\ecvATwo}{\ecvAOne}{\ecvATwo}{\ecvAOne}
  \ecvlanguagefooter[15pt]{(*)}

\ecvitem{Social skills and competences}{I regularly attend IT meetups, was an active member of the Bergamo Linux User Group (\url{http://bglug.it}) and helped start another small programming tech group: Codelovers (\url{http://codelovers.it})}

\end{europecv}
\end{document}
